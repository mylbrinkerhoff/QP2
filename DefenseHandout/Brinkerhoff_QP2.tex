% !TEX TS-program = lualatex
% !TEX encoding = UTF-8 Unicode

\documentclass[12pt, letterpaper]{article}

%%BIBLIOGRAPHY- This uses biber/biblatex to generate bibliographies according to the 
%%Unified Style Sheet for Linguistics
\usepackage[main=american, german]{babel}% Recommended
\usepackage{csquotes}% Recommended
\usepackage[backend=biber,
             style=unified,
             maxcitenames=3,
             maxbibnames=99,
             natbib,
             url=false]{biblatex}
\addbibresource{Library.bib}
\setcounter{biburlnumpenalty}{100}  % allow URL breaks at numbers
%\setcounter{biburlucpenalty}{100}   % allow URL breaks at uppercase letters
%\setcounter{biburllcpenalty}{100}   % allow URL breaks at lowercase letters

%%TYPOLOGY
\usepackage[svgnames]{xcolor} % Specify colors by their 'svgnames', for a full list of all colors available see here: http://www.latextemplates.com/svgnames-colors
%\usepackage[compact]{titlesec}
%\titleformat{\section}[runin]{\normalfont\bfseries}{\thesection.}{.5em}{}[.]
%\titleformat{\subsection}[runin]{\normalfont\scshape}{\thesubsection}{.5em}{}[.]
\usepackage[hmargin=1in,vmargin=1in]{geometry}  %Margins
\usepackage{graphicx} % 
\usepackage{stackengine} %Package to allow text above or below other text, Also helpful for HG weights 
\usepackage{fontspec} %Selection of fonts must be ran in XeLaTeX
\usepackage{amssymb} %Math symbols
\usepackage{amsmath} % Mathematical enhancements for LaTeX
\usepackage{setspace} %Linespacing
\usepackage{multicol} %Multicolumn text
\usepackage{enumitem} %Allows for continuous numbering of lists over examples, etc.
\usepackage{multirow} %Useful for combining cells in tablesbrew 
\usepackage{booktabs}
\usepackage{hanging}
\usepackage{fancyhdr} %Allows for the 
\pagestyle{fancy}
\fancyhead[L]{\textit{QP Defense Handout}} 
\fancyhead[R]{\textit{\today}} 
\fancyfoot[L,R]{} 
\fancyfoot[C]{\thepage} 
\renewcommand{\headrulewidth}{0.4pt}
\setlength{\headheight}{14.5pt} % ...at least 14.49998pt
% \usepackage{fourier} % This allows for the use of certain wingdings like bombs, frowns, etc.
% \usepackage{fourier-orns} %More useful symbols like bombs and jolly-roger, mostly for OT
\usepackage[colorlinks,allcolors={black},urlcolor={blue}]{hyperref} %allows for hyperlinks and pdf bookmarks
% \usepackage{url} %allows for urls
% \def\UrlBreaks{\do\/\do-} %allows for urls to be broken up
\usepackage[normalem]{ulem} %strike out text. Handy for syntax
\usepackage{tcolorbox}
\usepackage{datetime2}
\usepackage{caption}
\usepackage{subcaption}

%%FONTS
\setmainfont{Libertinus Serif}
\setsansfont{Libertinus Sans}
\setmonofont[Scale=MatchLowercase]{Libertinus Mono}

%%PACKAGES FOR LINGUISTICS
%\usepackage{OTtablx} %Generating tableaux with using TIPA
\usepackage[noipa]{OTtablx} % Use this one generating tableaux without using TIPA
%\usepackage[notipa]{ot-tableau} % Another tableau drawing packing use for posters.
% \usepackage{linguex} % Linguistic examples
% \usepackage{langsci-linguex} % Linguistic examples
\usepackage{langsci-gb4e} % Language Science Press' modification of gb4e
% \usepackage{langsci-avm} % Language Science Press' AVM package
\usepackage{tikz} % Drawing Hasse diagrams
% \usepackage{pst-asr} % Drawing autosegmental features
\usepackage{pstricks} % required for pst-asr, OTtablx, pst-jtree.
% \usepackage{pst-jtree} %Syntax tree draawing software
% \usepackage{tikz-qtree} % Another syntax tree drawing software. Uses bracket notation.
\usepackage[linguistics]{forest} % Another syntax tree drawing software. Uses bracket notation.
% \usepackage{ling-macros} % Various linguistic macros. Does not work with linguex.
% \usepackage{covington} % Another linguistic examples package.
\usepackage{leipzig} % Offers support for Leipzig Glossing Rules

%%LEIPZIG GLOSSING FOR ZAPOTEC
\newleipzig{el}{el}{elder} % Elder pronouns
\newleipzig{hu}{hu}{human} % Human pronouns
\newleipzig{an}{an}{animate} % Animate pronouns
\newleipzig{in}{in}{inanimate} % Inanimate pronouns
\newleipzig{pot}{pot}{potential} % Potential Aspect
\newleipzig{cont}{cont}{continuative} % Continuative Aspect
% \newleipzig{pot}{pot}{potential} % Potential Aspect
\newleipzig{stat}{stat}{stative} % Potential Aspect
\newleipzig{and}{and}{andative} % Andative Aspect
\newleipzig{ven}{ven}{venative} % Venative Aspect
% \newleipzig{res}{res}{restitutive} % Restitutive Aspect
\newleipzig{rep}{rep}{repetitive} % Repetitive Aspect

%%TITLE INFORMATION
\title{TITLE}
\author{Mykel Loren Brinkerhoff}
\date{\today}

%%MACROS
\newcommand{\sub}[1]{\textsubscript{#1}}
\newcommand{\supr}[1]{\textsuperscript{#1}}
\providecommand{\lsptoprule}{\midrule\toprule}
\providecommand{\lspbottomrule}{\bottomrule\midrule}
\newcommand{\fittable}[1]{\resizebox{\textwidth}{!}{#1}}

\makeatletter
\renewcommand{\paragraph}{%
  \@startsection{paragraph}{4}%
  {\z@}{0ex \@plus 1ex \@minus .2ex}{-1em}%
  {\normalfont\normalsize\bfseries}%
}
\makeatother
\parindent=10pt


\begin{document}

%%If using linguex, need the following commands to get correct LSA style spacing
%% these have to be after  \begin{document}
    % \setlength{\Extopsep}{6pt}
    % \setlength{\Exlabelsep}{9pt}%effect of 0.4in indent from left text edge
%%

%% Line spacing setting. Comment out the line spacing you do not need. Comment out all if you want single spacing.
%\doublespacing
%\onehalfspacing

\begin{center}
    {\Large \textbf{The acoustics of phonation in Santiago Laxopa Zapotec}}
    \vspace{6pt}

    Mykel Loren Brinkerhoff
\end{center}
%\maketitle
%\maketitleinst
\thispagestyle{fancy}

\tableofcontents

%------------------------------------
\section{Introduction} \label{sec:Introduction}
%------------------------------------

\begin{itemize}
    \item Phonation is a process where the larynx is used to alter the way different sounds are produced. 
	\item This use of the larynx produces sounds which vary from being more breathy or creaky.
	\begin{itemize}
		\item Although other types of phonation also exist (see \cite{eslingVoiceQualityLaryngeal2019} for a detailed discussion on the different phonation types that exist and how the larynx produces them).
	\end{itemize} 
	\item In some languages these alterations are described as being pathological, with some speakers just being more breathy or creaky than others (e.g., \cite{klattAnalysisSynthesisPerception1990}). 
    \item In other languages these phonation Sometimes these different phonation contrasts are used phonemically.
    \item This paper explores these 
\end{itemize}

%------------------------------------
\section{Background} \label{sec:Background}
%------------------------------------

\begin{itemize}
	\item Since 
	\item \citet{ladefogedPreliminariesLinguisticPhonetics1971,gordonPhonationTypesCrosslinguistic2001}
\end{itemize}

\begin{figure}[!ht]
	\centering
	\includegraphics[width=.6\textwidth]{../Phonation.png}
	\caption{Simplified one-deminsional model for phonation. Based on \citet{ladefogedPreliminariesLinguisticPhonetics1971,gordonPhonationTypesCrosslinguistic2001}}.
	\label{fig:Phonation}
\end{figure}
%------------------------------------
\section{Santiago Laxopa Zapotec} \label{sec:SLZ}
%------------------------------------

\begin{itemize}
    \item Santiago Laxopa Zapotec (SLZ), endonym \textit{Dille'xhunh Laxup}, is a a Northern Zapotec language spoken by approximately 1000 people in the municipality of Santiago Laxopa, Ixtlán, Oaxaca, Mexico and in diaspora communities in Mexico and the United States \citep{adlerAcousticsPhonationTypes2016,adlerDerivationVerbInitiality2018,foleyForbiddenCliticClusters2018,foleyExtendingPersonCaseConstraint2020}.
    \item Closely related to San Bartolomé Zoogocho Zapotec \citep{longDiccionarioZapotecoSan2005,sonnenscheinDescriptiveGrammarSan2005} and shares a high level of mutual intelligibility with it.
    \item SLZ is similar to other Zapotecan languages in distinguishing lenis and fortis consonants \citep[e.g.,][]{nellisFortisLenisCajonos1980,jaegerFortisLenisQuestion1983,uchiharaFortisLenisGlides2016}.
\end{itemize}

\begin{table}[!h]
	\centering
	\caption{Consonant inventory for Santiago Laxopa Zapotec}
	\label{tab:SLZcons}
	\fittable{
	\begin{tabular}{llcccccccc}
	\lsptoprule
		  &  & bilabial & alveolar  & post- & retroflex & palatal &velar &labio-  &  uvular \\
		 &&&&alveolar&  &&&velar& \\
	\midrule
	stop 		& lenis   & b  & d  & & & & g & gʷ & \\
				& fortis  & p  & t  & & & & k & kʷ & \\
	fricative   & lenis   &    & z  & ʒ & ʐ &  & &  & ʁ \\
		        & fortis  &    & s  & ʃ & ʂ & ç & & & \\
	affricate 	& lenis   &    & d͡z & & & & & & \\
				& fortis  &    & t͡s & & t͡ʃ & & & & \\
    nasal    	& lenis   &	   & n  & & & & & & \\
				& fortis  &	mː & nː & & & & & & \\
	lateral  	& lenis   &    & l & & & & & & \\
				& fortis  &    & lː & & & & & & \\
	trill		& 		  &    & r & & &  & &  & \\ 			
	approximate & 		  &    & & & & & & w & \\ 
	\lspbottomrule
	\end{tabular}
	}
\end{table}

\begin{itemize}
    \item SLZ has a standard five vowel inventory. 
\end{itemize}

\begin{table}[!h]
	\centering
	\caption{Vowel qualities in Santiago Laxopa Zapotec.}
    \label{tab:SLZvowels}
	\begin{tabular}{lccc}
	\lsptoprule
	&  front& central  & back \\
	\midrule
	high   	&  i  &     &   u \\
	mid    	&  e  &   	& 	o \\
	low   	&     &  a 	&	  \\
	\lspbottomrule
	\end{tabular}
\end{table}

\begin{itemize}
	\item These five vowels, additionally, appear with one of four different phonation types which will be discussed in greater detail in Section~\ref{sec:Phonation}.
\end{itemize}

%------------------------------------
\subsection{Tone in Santiago Laxopa Zapotec} \label{sec:Tone}
%------------------------------------

\begin{itemize}
    \item Similar to other Otomanguean languages, SLZ is tonal\citep{suarezMesoamericanIndianLanguages1983,campbellMesoAmericaLinguisticArea1986,silvermanLaryngealComplexityOtomanguean1997,campbellOtomangueanHistoricalLinguistics2017a,campbellOtomangueanHistoricalLinguistics2017}.
    \item SLZ has five distinct tonal patterns that appear on the syllables of nouns, see Table~\ref{tab:tones}. 
\end{itemize}

\begin{table}[!h]
	\centering
	\caption{Examples of the five tonal patterns observed in the Santiago Laxopa Zapotec words.}
	\label{tab:tones}
	\begin{tabular}{lllll}
	\lsptoprule
	High   	&  a\supr{H}  &  \textit{xha}   &  [ ʐa\supr{H} ] & `clothing.\textsc{poss}'\\
	Mid    	&  a\supr{M}  &  \textit{lhill} 	& [ liʒ\supr{M} ] & `house.\textsc{poss}' \\
	Low   	&  a\supr{L}  &  \textit{yu'} 	&	 [ çuˀ\supr{L} ] & `earth'\\
	Rising	&  a\supr{MH}  &  \textit{yu'u} 	&	[ çuˀu\supr{MH} ] & `quicklime (Sp. cal)' \\
	Falling &  a\supr{HL}  &  \textit{yu'u}  &	[çuˀu\supr{HL}] &	`house' \\
	\lspbottomrule
	\end{tabular}
\end{table}

\begin{itemize}
	\item These five tonal patterns are illustrated in Figures~\ref{fig:FSRTonePlot} and \ref{fig:RDTonePlot} for two different SLZ speakers. 
	\item Figures~\ref{fig:FSRTonePlot} and \ref{fig:RDTonePlot} shows the five tonal contrasts averaged for each tonal contrast from the onset to ending of the vowel. 
	\item We can ignore the first 20-25\% of the measure due to the influence of transitions out of the consonantal onsets. 
\end{itemize}

\begin{figure}[!ht]
	\centering
	\includegraphics[width=0.9\textwidth]{../FSRTonePlot.png}
	\caption{Tonal contrasts for FSR averaged and time normalized. Each line in this graph represents the average of approximately 10 syllables for each tonal pattern. }
	\label{fig:FSRTonePlot}
\end{figure}

\begin{figure}[!ht]
	\centering
	\includegraphics[width=0.9\textwidth]{../RDTonePlot.png}
	\caption{Tonal contrasts for RD averaged and time normalized. Each line in this graph represents the average of approximately 10 syllables for each tonal pattern.}
	\label{fig:RDTonePlot}
\end{figure}
%------------------------------------
\subsection{Phonation in Santiago Laxopa Zapotec} \label{sec:Phonation}
%------------------------------------

\begin{itemize}
	\item Zapotecan languages commonly make use of contrastive phonation on vowels \citep[e.g.,][]{avelinobecerraTopicsYalalagZapotec2004,longDiccionarioZapotecoSan2005,avelinoAcousticElectroglottographicAnalyses2010,lopeznicolasEstudiosFonologiaGramatica2016,chavez-peonInteractionMetricalStructure2010}.
	\item SLZ is no different and has four contrastive phonation types: modal /a/, breathy /a̤/, checked /aˀ/, and laryngealized /aˀa/. 
	\item 
\end{itemize}

\ea \label{ex:YA} Four-way near minimal phonation contrast
	\ea \textit{yag}  [çag\supr{L}] `tree; wood; almúd (unit of measurement approximately 4kg)'
	\ex \textit{yah}  [ça̤\supr{L}] `metal; rifle; bell'
	\ex \textit{yu'}  [çuˀ\supr{L}]  `earth'
	\ex \textit{ya'a}  [çaˀa\supr{L}]  `market'
	\z 
\z 

\begin{figure}[!h]
	\centering
	\includegraphics[width=0.9\textwidth]{../yah.png}
	\caption{FSR's breathy vowel in the word \textit{yah} `metal; rifle'}
	\label{fig:BreathyVowel}
\end{figure}

\begin{figure}[!h]
	\centering
	% [INSERT YA SPECTROGRAM AND WAVEFORM]
	\includegraphics[width=0.9\textwidth]{../RD_yu'.png}
	\caption{RD's checked vowel in the word \textit{yu'} `earth'}
	\label{fig:CheckedVowel}
\end{figure}

\begin{table}[!h]
	\centering
	\caption{Layngealized Vowels in Yalálag Zapotec}
	\label{tab:laryngeal}
	 \begin{tabular}{ll}
	\lsptoprule
	/VˀV/	&  [VʔV]  \\
			&  [VV̰V]   \\
			&  [VV̰ːV̆]  \\
			&  [VV̰V̰]	\\
	\lspbottomrule
	\end{tabular}
\end{table}

\begin{figure}[!h]
	\centering
	\begin{subfigure}{.5\textwidth}
		\centering
		\includegraphics[width=\linewidth]{../za'a.png}
		\caption{\textit{za'a} `corncob'}
		\label{fig:za'a}
	\end{subfigure}%
	\begin{subfigure}{.5\textwidth}
		\centering
		\includegraphics[width=\linewidth]{../xa'ag.png}
		\caption{\textit{xa'ag} `topil'}
		\label{fig:xa'ag}
	\end{subfigure}	
	\caption{Comparison of FSR's laryngealized vowels in \textit{za'a} `corncob' and \textit{xa'ag} `topil'}
	\label{fig:FSRLaryngeal}
\end{figure}

\begin{figure}[!h]
	\centering
	\begin{subfigure}{.5\textwidth}
		\centering
		\includegraphics[width=\linewidth]{../RD_za'a.png}
		\caption{\textit{za'a} `corncob'}
		\label{fig:za'a}
	\end{subfigure}%
	\begin{subfigure}{.5\textwidth}
		\centering
		\includegraphics[width=\linewidth]{../RD_xa'ag.png}
		\caption{\textit{xa'ag} `topil'}
		\label{fig:xa'ag}
	\end{subfigure}
	\caption{Comparison of RD's laryngealized vowels in \textit{za'a} `corncob' and \textit{xa'ag} `topil'}
	\label{fig:RDLaryngeal}
\end{figure}

%------------------------------------
\subsection{Interaction of tone and phonation in Santiago Laxopa Zapotec} \label{sec:TonePhonation}
%------------------------------------

\begin{table}[!ht]
	\centering
	\caption{SLQZ tone and phonation interactions \citep{chavez-peonInteractionMetricalStructure2010}.}
	\label{tab:SLQZ}
	 \begin{tabular}{lcccc}
	  \lsptoprule
					  &	 \textbf{Modal}  & \textbf{Breathy} & \textbf{Creaky} & \textbf{Interrupted} \\
		  High	& ✔︎ & -- & ✔︎ & ✔︎ \\
		  Low & ✔︎ & ✔︎ & ✔︎ & ✔︎ \\
		  Falling & ✔︎ & ✔︎ & ✔︎ & ✔︎ \\
		  Rising & ✔︎ & -- & -- & -- \\
	  \lspbottomrule
	 \end{tabular}
\end{table}

\begin{table}[!h]
	\caption{Number of unique syllables for each interaction of tone and phonation in the data.}
	\label{tab:ToneVoiceQuality}
	\centering

	\begin{tabular}{lcccc}
	\lsptoprule
		& \textbf{Modal} & \textbf{Breathy} & \textbf{Checked} & \textbf{Laryngealized} \\
	\hline
	High		& 12 & -- & 5	& 3 \\
	Mid			& 7 & 1  & 2	& 1 \\
	Low			& 17 & 5  & 9	& 3 \\
	High-Low	& 9 & 1  & 2	& 1 \\
	Mid-High	& 1	 & 1  & --	& 1 \\
	\lspbottomrule
	\end{tabular}
\end{table}


%------------------------------------
\section{Methodology} \label{sec:Methods}
%------------------------------------

\begin{figure}[!h]
	\centering
	\includegraphics[width=0.9\textwidth]{../Harmonics.png}
	\caption{Spectral slice with LPC smoothed line overlaid for the vowel [e]. The harmonics in the spectral slice are represented by each of the dark peaks. The leftmost black solid line peak is the first harmonic (H1) and each subsequent peak represents the next highest harmonic (H2 through H\textit{n}). The red dotted line represents an LPC smoothed line which identifies the formants by the peaks. Each of the harmonics that are closest to the formant peak is identified as A1 through A\textit{n}.}
	\label{fig:Harmonics}
\end{figure}

%------------------------------------
\section{Results} \label{sec:Results}
%------------------------------------

%------------------------------------
\subsection{H1-H2 spectral-tilt} \label{sec:H1H2}
%------------------------------------


\begin{figure}[!ht]
	\centering
	\begin{subfigure}{.5\textwidth}
		\centering
		\includegraphics[width=0.9\textwidth]{../mean_FSR_h1h2_1st.png}
		\caption{FSR's H1-H2 values.}
		\label{fig:FSRh1h2first} 
	\end{subfigure}%
	\begin{subfigure}{.5\textwidth}
		\centering
		\includegraphics[width=0.9\textwidth]{../mean_RD_h1h2_1st.png}
		\caption{RD's H1-H2 values.}
		\label{fig:RDh1h2first} 
	\end{subfigure}
	\caption{Mean H1-H2 values for the first third of the vowel according to each phonation type. }
	\label{fig:h1h2first}
\end{figure}

\begin{figure}[!ht]
	\centering
	\begin{subfigure}{.5\textwidth}
		\centering
		\includegraphics[width=0.9\textwidth]{../mean_FSR_h1h2_2nd.png}
		\caption{FSR's H1-H2 values.}
		\label{fig:FSRh1h2second} 
	\end{subfigure}%
	\begin{subfigure}{.5\textwidth}
		\centering
		\includegraphics[width=0.9\textwidth]{../mean_RD_h1h2_2nd.png}
		\caption{RD's H1-H2 values.}
		\label{fig:RDh1h2second} 
	\end{subfigure}
	\caption{Mean H1-H2 values for the second third of the vowel according to each phonation type.}
	\label{fig:h1h2second}
\end{figure}

\begin{figure}[!ht]
	\centering
	\begin{subfigure}{.5\textwidth}
		\centering
		\includegraphics[width=0.9\textwidth]{../mean_FSR_h1h2_3rd.png}
		\caption{FSR's H1-H2 values.}
		\label{fig:FSRh1h2third} 
	\end{subfigure}%
	\begin{subfigure}{.5\textwidth}
		\centering
		\includegraphics[width=0.9\textwidth]{../mean_RD_h1h2_3rd.png}
		\caption{RD's H1-H2 values.}
		\label{fig:RDh1h2third} 
	\end{subfigure}
	\caption{Mean H1-H2 values for the final third of the vowel according to each phonation type. }
	\label{fig:h1h2third}
\end{figure}
%------------------------------------
\subsection{H1-A3 spectral-tilt} \label{sec:H1A3}
%------------------------------------

\begin{figure}[!h]
	\centering
	\begin{subfigure}{.5\textwidth}
		\centering
		\includegraphics[width=0.9\textwidth]{../mean_FSR_h1a3_First.png}
		\caption{FSR's H1-A3 values.}
		\label{fig:FSRh1a3first} 
	\end{subfigure}%
	\begin{subfigure}{.5\textwidth}
		\centering
		\includegraphics[width=0.9\textwidth]{../mean_RD_h1a3_First.png}
		\caption{RD's H1-A3 values.}
		\label{fig:RDh1a3first} 
	\end{subfigure}
	\caption{H1-A3 values for FSR (a) and RD (b) for the first third of the vowel. }
	\label{fig:h1a3first}
\end{figure}

In the first third of the vowel, RD's mean value for H1-A3 is lower than the modal's H1-A3 value. However, there is a large degree of overlap between modals, checked, and laryngealized H1-A3 values, as evidenced by the boxes covering the same regions, see Figure~\ref{fig:h1a3first}. 

\begin{figure}[!h]
	\centering
	\begin{subfigure}{.5\textwidth}
		\centering
		\includegraphics[width=0.9\textwidth]{../mean_FSR_h1a3_Second.png}
		\caption{FSR's H1-A3 values.}
		\label{fig:FSRh1a3second} 
	\end{subfigure}%
	\begin{subfigure}{.5\textwidth}
		\centering
		\includegraphics[width=0.9\textwidth]{../mean_RD_h1a3_Second.png}
		\caption{RD's H1-A3' values.}
		\label{fig:RDh1a3second} 
	\end{subfigure}
	\caption{H1-A3 values for FSR (a) and RD (b) for the second third of the vowel. }
	\label{fig:h1a3second}
\end{figure}

In the second third of the vowel, Figure~\ref{fig:h1a3second}, the breathy vowel continues to be higher than the modal vowel. The checked and laryngealized vowels H1-A3 values for FSR are uninformative because of the large degree of overlap. For RD, these same measurements show a lower H1-A3 value than the modals which is consistent with creakier productions of vowels. This lower H1-A3 continues throughout the rest of the vowel for laryngealized vowels, see Figure~\ref{fig:RDh1a3third}. This behavior is consistent with the observation that RD performs creaky voice throughout their production of laryngealized vowels. For the checked vowels, the measurements are very similar to those of the modal vowel. 

In looking at the final portion of the vowels, Figure~\ref{fig:h1a3third}, the measurements continue to show similar behavior to the second portion for both FSR and RD. However, one exception is the lower H1-A3 value for FSR's checked vowels, suggesting that FSR produces a period of creakiness in the last portion of the vowel.

\begin{figure}[!ht]
	\centering
	\begin{subfigure}{.5\textwidth}
		\centering
		\includegraphics[width=0.9\textwidth]{../mean_FSR_h1a3_third.png}
		\caption{FSR's H1-A3 values.}
		\label{fig:FSRh1a3third} 
	\end{subfigure}%
	\begin{subfigure}{.5\textwidth}
		\centering
		\includegraphics[width=0.9\textwidth]{../mean_RD_h1a3_third.png}
		\caption{RD's H1-A3 values.}
		\label{fig:RDh1a3third} 
	\end{subfigure}
	\caption{H1-A3 values for FSR (a) and RD (b) for the final third of the vowel. }
	\label{fig:h1a3third}
\end{figure}

%------------------------------------
\subsection{Cepstral Peak Prominence} \label{sec:CPP}
%------------------------------------

\begin{figure}[!ht]
	\centering
	\begin{subfigure}{.5\textwidth}
		\centering
		\includegraphics[width=0.9\textwidth]{../mean_FSR_cpp_First.png}
		\caption{FSR's CPP values.}
		\label{fig:FSRcppfirst} 
	\end{subfigure}%
	\begin{subfigure}{.5\textwidth}
		\centering
		\includegraphics[width=0.9\textwidth]{../mean_RD_cpp_First.png}
		\caption{RD's CPP values.}
		\label{fig:RDcppfirst} 
	\end{subfigure}
	\caption{CPP values for FSR (a) and RD (b) for the first third of the vowel. }
	\label{fig:cppfirst}
\end{figure}

\begin{figure}[!ht]
	\centering
	\begin{subfigure}{.5\textwidth}
		\centering
		\includegraphics[width=0.9\textwidth]{../mean_FSR_cpp_Second.png}
		\caption{FSR's CPP values.}
		\label{fig:FSRcppsecond} 
	\end{subfigure}%
	\begin{subfigure}{.5\textwidth}
		\centering
		\includegraphics[width=0.9\textwidth]{../mean_RD_cpp_Second.png}
		\caption{RD's CPP values.}
		\label{fig:RDcppsecond} 
	\end{subfigure}
	\caption{CPP values for FSR (a) and RD (b) for the second third of the vowel. }
	\label{fig:cppsecond}
\end{figure}

\begin{figure}[!ht]
	\centering
	\begin{subfigure}{.5\textwidth}
		\centering
		\includegraphics[width=0.9\textwidth]{../mean_FSR_cpp_third.png}
		\caption{FSR's CPP values.}
		\label{fig:FSRcppthird} 
	\end{subfigure}%
	\begin{subfigure}{.5\textwidth}
		\centering
		\includegraphics[width=0.9\textwidth]{../mean_RD_cpp_third.png}
		\caption{RD's CPP values.}
		\label{fig:RDcppthird} 
	\end{subfigure}
	\caption{CPP values for FSR (a) and RD (b) for the final third of the vowel. }
	\label{fig:cppthird}
\end{figure}

%------------------------------------
\subsection{Statistical results} \label{sec:Stats}
%------------------------------------

\begin{table}[!h]
	\centering
	\caption{Results of the mixed-effects linear regression analysis on the first third of the vowel for H1-H2. }
	\label{tab:H1H2_First}
	 \begin{tabular}{llllll}
	  \lsptoprule
						&  Estimate  & Std. Error & df & t value & p-value \\
	  	Breathy   		&  -0.3049  &   0.5795 & 116.4255 &  -0.526  &  0.600 \\
		Checked    		&  -0.4731  &   0.4029 & 145.6184 &  -1.174  &  0.242 \\
		Laryngealized	&  -0.2938  &   0.4675 & 101.9977 &  -0.629  &  0.531 \\
	  \lspbottomrule
	 \end{tabular}
\end{table}

As can be recalled from Section~\ref{sec:H1H2}, both FSR and RD show a lower value for H1-H2 for breathy vowels in the last two-thirds of the vowels when compared to the model vowel's H1-H2 values. The results of the statistical analysis for the last two-thirds of the vowel, presented in Table~\ref{tab:H1H2_Second} and Table~\ref{tab:H1H2_Third}, show that this behavior is significant.

At no point do the other phonation types reach significance with respect to H1-H2. 

\begin{table}[!h]
	\centering
	\caption{Results of the mixed-effects linear regression analysis on the second third of the vowel for H1-H2. }
	\label{tab:H1H2_Second}
	 \begin{tabular}{llllll}
	  \lsptoprule
						&  Estimate  & Std. Error & df & t value & p-value \\
	  	Breathy   		&  -3.4158  &   1.3400 & 126.9590 &  -2.549  &  0.0120 \\
		Checked    		&  -1.7466  &   0.9197 & 146.9995 &  -1.899  &  0.0595 \\
		Laryngealized	&  -0.7405  &   1.0852 & 117.0669 &  -0.682  &  0.4963 \\
	  \lspbottomrule
	 \end{tabular}
\end{table}

\begin{table}[!h]
	\centering
	\caption{Results of the mixed-effects linear regression analysis on the final third of the vowel for H1-H2. }
	\label{tab:H1H2_Third}
	 \begin{tabular}{llllll}
	  \lsptoprule
						&  Estimate  & Std. Error & df & t value & p-value \\
	  	Breathy   		&  -2.2131  &   1.0291 & 125.2092 &  -2.151  &  0.0334 \\
		Checked    		&  -1.3487  &   0.6952 & 146.2461 &  -1.940  &  0.0543 \\
		Laryngealized	&  -0.8676  &   0.8202 & 116.8963 &  -1.058  &  0.2923 \\
	  \lspbottomrule
	 \end{tabular}
\end{table}

The second statistical analysis for the H1-A3 measurement shows some very clear behavior for breathy voice. As can be recalled from Section~\ref{sec:H1A3}, we observe that breathy voice is clearly identified in all portions of the vowel with an elevated H1-A3 value when compared to the model vowels' values. We see that at all portions of the vowel, as seen in Tables~\ref{tab:H1A3_First}, \ref{tab:H1A3_Second}, \ref{tab:H1A3_Third} show significance.  

\begin{table}[!h]
	\centering
	\caption{Results of the mixed-effects linear regression analysis on the first third of the vowel for H1-A3. }
	\label{tab:H1A3_First}
	 \begin{tabular}{llllll}
	  \lsptoprule
						&  Estimate  & Std. Error & df & t value & p-value \\
	  	Breathy   		&  4.29036  &  1.31288 & 137.12067 &  3.268  &  0.00137\\
		Checked    		&  0.15001  &  0.84476 & 134.62215 &  0.178  &  0.85932 \\
		Laryngealized	& -0.05694  &  1.07130 & 137.16596 & -0.053  &  0.95769\\
	  \lspbottomrule
	 \end{tabular}
\end{table}

\begin{table}[!h]
	\centering
	\caption{Results of the mixed-effects linear regression analysis on the second third of the vowel for H1-A3. }
	\label{tab:H1A3_Second}
	 \begin{tabular}{llllll}
	  \lsptoprule
						&  Estimate  & Std. Error & df & t value & p-value \\
	  	Breathy   		&  4.398    &  2.027 & 147.003 &  2.169 &  0.0317 \\
		Checked    		&  1.195    &  1.426 & 147.000 &  0.838 &  0.4033  \\
		Laryngealized	& -2.694    &  1.627 & 147.001 & -1.656 &  0.0998 \\
	  \lspbottomrule
	 \end{tabular}
\end{table}

\begin{table}[!h]
	\centering
	\caption{Results of the mixed-effects linear regression analysis on the final third of the vowel for H1-A3. }
	\label{tab:H1A3_Third}
	 \begin{tabular}{llllll}
	  \lsptoprule
						&  Estimate  & Std. Error & df & t value & p-value \\
	  	Breathy   		&  5.2928   &  1.7966 & 127.0400 &  2.946 & 0.00383 \\
		Checked    		& -0.7825   &  1.2487 & 147.1011 & -0.627 & 0.53189 \\
		Laryngealized	& 0.2669    &  1.4224 & 117.4917 &  0.188 & 0.85151 \\
	  \lspbottomrule
	 \end{tabular}
\end{table}

However, the other phonations failed to reach significance when evaluated against H1-A3.

It is important to realize that at this point that H1-H2 and H1-A3 both have failed to account for checked and laryngealized phonations. This is also born out by the observations in Sections~\ref{sec:H1H2} and \ref{sec:H1A3} which did not show any remarkable differences from the model. When we consider CPP we see both checked and laryngealized vowels clearly differentiated. This is born out in the statistical analysis. This analysis took CPP as fixed and speaker and word as random. This analysis showed that the first third of the vowel, Table~\ref{tab:CPP_First} both breathy and checked voice were identifiable with CPP. 

% Based on the results of the statistical analysis, CPP is informative for differentiating checked from laryngealized voice. Laryngealized voice shows a statistically significance in the second-third of the vowel. This is consistent with the observed behavior approximately in the center of the vowel. Despite the differences in how the the laryngealized vowels are produced between the speakers, there is a difference in the CPP value in the middle of the vowel it 

\begin{table}[!h]
	\centering
	\caption{Results of the mixed-effects linear regression analysis on the first third of the vowel for CPP. }
	\label{tab:CPP_First}
	 \begin{tabular}{llllll}
	  \lsptoprule
						&  Estimate  & Std. Error & df & t value & p-value \\
	  	Breathy   		&  3.4470   &  1.1743 & 145.9760 &  2.935 & 0.00387 \\
		Checked    		& -1.4190   &  0.7088 & 120.5020 & -2.002 & 0.04754 \\
		Laryngealized	&  0.8240   &  0.9530 & 147.0083 &  0.865 & 0.38868 \\
	  \lspbottomrule
	 \end{tabular}
\end{table}

When considering the second-third of the vowel, Table~\ref{tab:CPP_Second}, laryngealized vowels become clearly identifiable by CPP which is born out by the significance of laryngealized phonation and the lack of significance with breathy and checked. This is consistent with the observation that somewhere in the middle of the vowel there is either a full glottal stop or a period of creakiness in the two speakers that were evaluated for this study. This observation also bears witness to observations made by \citet{avelinoAcousticElectroglottographicAnalyses2010} about the how each of the different manners that laryngealized vowels are produced in Yalálag Zapotec each have a period of model phonation followed by a aperiodicity or a glottal constriction beginning in the middle of the vowel. 

\begin{table}[!h]
	\centering
	\caption{Results of the mixed-effects linear regression analysis on the second third of the vowel for CPP. }
	\label{tab:CPP_Second}
	 \begin{tabular}{llllll}
	  \lsptoprule
						&  Estimate  & Std. Error & df & t value & p-value \\
	  	Breathy   		&  1.5044   &  1.4073 & 128.0083 &  1.069 & 0.287094 \\
		Checked    		& -0.5804   &  0.9789 & 146.0792 & -0.593 & 0.554154 \\
		Laryngealized	& -2.2898   &  1.1354 & 117.6213 & -2.017 & 0.046006 \\
	  \lspbottomrule
	 \end{tabular}
\end{table}

In considering the final portion of the vowel, Table~\ref{tab:CPP_Third}, the statistical analysis shows that only checked vowels can be reliably determined using CPP. This observations is consistent with the facts that checked phonation consists of a period of creakiness or full glottal closure at the end of the vowel. The other phonation types fail to reach significance.

\begin{table}[!h]
	\centering
	\caption{Results of the mixed-effects linear regression analysis on the final third of the vowel for CPP. }
	\label{tab:CPP_Third}
	 \begin{tabular}{llllll}
	  \lsptoprule
						&  Estimate  & Std. Error & df & t value & p-value \\
	  	Breathy   		&  1.6391   &  1.0173  & 139.8563 &  1.611 & 0.10939 \\
		Checked    		& -2.1386   &  0.6449  & 129.1392 & -3.316 & 0.00119 \\
		Laryngealized	& -1.1158   &  0.8142  & 140.5570 & -1.370 & 0.17274\\
	  \lspbottomrule
	 \end{tabular}
\end{table}

%------------------------------------
\section{Discussion} \label{sec:Discussion}
%------------------------------------

%------------------------------------
\subsection{Laryngeal Complexity Hypothesis} \label{sec:LCH}
%------------------------------------

\begin{itemize}
    \item \citet{silvermanLaryngealComplexityOtomanguean1997,blankenshipTimeCourseBreathiness1997,blankenshipTimingNonmodalPhonation2002}
\end{itemize}

%------------------------------------
\subsection{Laryngeal Articulator Model} \label{sec:LAM}
%------------------------------------

\begin{table}[!ht]
	\centering
	\caption{A list of the different nodes and their abbreviations in the Laryngeal Articulator Model.}
	\label{tab:States}
	 \begin{tabular}{ll}
	  \lsptoprule
	  States/Nodes	&	 Physiological description \\
	  \hline
	  	vfo   	&  vocal folds open (abducted) \\
		vfc    	&  vocal folds closed (adducted/prephonation)\\
		epc   	&  epilaryngeal constriction\\
		epv			&  epilaryngeal vibration \\
		tfr 		&  tongue fronting \\
		tre 		&  tongue retraction \\
		tra 		&  tongue raising \\
		tdb 		&	 tongue double bunching\\
		↑lx     &  raised larynx\\
		↓lx			&  lowered larynx\\
		Hf0			&  increased vocal fold tension, less vibrating mass (high f0)\\
		Lf0			&  decreased vocal fold tension, more vibrating mass (lower f0)\\
	  \lspbottomrule
	 \end{tabular}
\end{table}

These twelve nodes not only represent the interactions of the larynx but also represents actual physiological representations. This means that any given node represents what is occurring with a given part of the larynx. For example, the node `epc'  represents any epilaryngeal constriction when activated. Now these nodes are not just independent entities but interact in complex ways with other nodes. These interactions are best captured as a network or web of nodes as seen in Figure~\ref{fig:LAMNetwork}. 

\begin{figure}[!ht]
	\centering
	\includegraphics[width=0.9\textwidth]{../LAMNetwork.png}
	\caption{Network of synergistic and anti-synergistic nodes in the Laryngeal Articulator Model. Taken from \citet{eslingVoiceQualityLaryngeal2019}.}
	\label{fig:LAMNetwork}
\end{figure}

%------------------------------------
\section{Conclusion} \label{sec:Conclusion}
%------------------------------------

%------------------------------------
%\subsection{} \label{}
%------------------------------------

%------------------------------------
%BIBLIOGRAPHY
%------------------------------------

%\singlespacing
% \nocite{*}
\printbibliography[heading=bibintoc]

\end{document} 